\section{Experimental Evaluation}\label{Se:experiments}

In this section, we describe the \Tool\ implementation, and present two sets of experiments that we have conducted to evaluate our approach.

\subsection{The \Tool\ System}\label{Se:impl}

\paragraph{Implementation} Similarly to existing tools like TaintDroid, \Tool\ is implemented as an instrumented version of the Android SDK. Specifically, we have instrumented version 4.1.1\_r6 of the SDK, which was chosen intentionally to match the most recent version of TaintDroid.\footnote{
	\href{http://appanalysis.org/download.html}{http://appanalysis.org/download.html}
} The experimental data we present indeed utilizes TaintDroid for tag propagation (as required for accurate resolution for relevant values).

Beyond the TaintDroid instrumentation scheme, the \Tool\ scheme specifies additional behaviors for sources and sinks within the SDK. At source points, a hook is added to record the private value read by the source statement (which acts as a reference value). At sink points, a hook is installed to apply Bayesian reasoning regarding the legitimacy of the sink.

Analogously to TaintDroid, \Tool\ performs privacy monitoring over APIs for file-system access and manipulation, inter-application and socket communication, reading the phone's state and location, and sending of text messages. \Tool\ also monitors the HTTP interface, camera, microphone, bluetooth and contacts. \OC{As explained in \secref{pseudo}, each of the privacy sources monitored by \Tool\ is mirrored by a tag/feature. The full list of features is as follows: \emph{IMEI}, \emph{IMSI}, \emph{AndroidID}, \emph{Location}, \emph{Microphone}, \emph{Bluetooth}, \emph{Camera}, \emph{Contacts} and \emph{FileSystem}.}

The \Tool\ implementation is configurable, enabling the user to switch between distance metrics as well as enable/disable information-flow tracking for precise/heuristic determination of relevant values. (See \secref{points}.) In our experiments, we tried both the Levenshtein and the Hamming metrics, but found no observable differences, and so we report the results only once. Our reasoning for why the metrics are indistinguishable is because we apply both to equal-length strings (see \secref{similarstr}), and have made sure to apply the same metric both offline and online, and so both metrics achieve a very similar effect in the Bayesian setting.

\paragraph{Training} To instantiate \Tool\ with the required estimates, as explained in \secref{estprob}, we applied the following methodology:
%
First, to estimate $\Pr(\emph{legitimate})$, we relied on (i) an extensive study by Hornyack et al. spanning 1,100 top-popular free Android apps~\cite{HHJSW:CCS11}, as well as (ii) a similarly comprehensive study by Enck et al.~\cite{EOMC:SEC11}, which also spans a set of 1,100 free apps. According to the data presented in these studies, approximately one out of three release points is illegitimate, and thus $\widehat{\Pr}(\emph{legitimate})=0.66$ and complementarily
$\widehat{\Pr}(\emph{illegitimate}) = 1-0.66 \approx 0.33$.

For the conditional probabilities $\widehat{\Pr}(X_i = x_{ij} | Y=y_k)$,  we queried Google Play for the 100 most popular apps (across all domains) in the geography of one of the authors. We then selected at random 35 of these apps, and analyzed their information-release behavior using debug breakpoints
(which we inserted via the {\tt adb} tool that is distributed as part of the Android SDK).

Illegitimate leaks that we detected offline mainly involved (i) location information and (ii) device and user identifiers, which is consistent with the findings reported by past studies~\cite{HHJSW:CCS11,EOMC:SEC11}. We confirmed that illegitimate leaks are largely correlated with high similarity between private data and sink arguments, and so we fixed six distance levels for each private item: $[0,4]$ and ``$\geq 5$''. (See \secref{featext}.) Finally, to avoid zero estimates for conditional probabilities while also minimizing data perturbation, we set the ``smoothing'' factor $l$ in \equref{smoothEstimate} at 1, where the illegitimate flows we detected were in the order of several dozens per private item.

\subsection{Experimental Hypotheses}

%We conducted all of our experiments on clean virtual-machine images with 4GB of RAM running version 12.04.2 LTS of the Ubuntu Linux distribution. For our experiments, we used the Android mobile device emulator included with the Android SDK. We loaded the TaintDroid and \Tool\ images into separate emulators running on different clones of the virtual machine. For TaintDroid, we followed the installation and setup steps provided online.

In our experimental evaluation of \Tool, we tested two hypotheses:
\begin{compactenum}
\item \underline{H1: Accuracy.}
    Bayesian reasoning, as implemented in \Tool,
     yields a significant improvement in leakage-detection accuracy compared to the baseline of information-flow tracking.
\item \underline{H2: Applicability.} For real-life applications, \Tool\ remains effective under relaxation of the 
     tag-based method for detection of relevant values and its stability improves.
\end{compactenum}

\subsection{H1: Accuracy \JCO{}{of \Tool}}
\label{Se:overhead}
\JC{To assess the accuracy of \Tool, we compared it to that of TaintDroid, a state-of-the-art information-flow tracking tool for privacy enforcement. Our experimental settings and results are described below.

\paragraph{Subjects} We applied both TaintDroid and \Tool\ to DroidBench, an independent and publicly available collection of benchmarks serving as testing ground for both static and dynamic privacy enforcement algorithms.} DroidBench models a large set of realistic challenges in leakage detection, including precise tracking of sensitive data through containers, handling of callbacks, field and object sensitivity, lifecycle modeling, inter-app communication, reflection and implicit flows.

\JC{
The DroidBench suite consists of 50 cases. We excluded from our experiment (i) 8 benchmarks that crash at startup, as well as (ii) 5 benchmarks that leak data via callbacks that we did not manage to trigger (e.g., {\tt onLowMemory()}),
as both TaintDroid and \Tool\ were naturally unable to detect leakages in these two cases.
%{, for a total of 37 benchmarks. For both of these categories, both TaintDroid and \Tool\ were naturally unable to detect leakages.}
The complete list of benchmarks that we used can be found in \tableref{accuracyDBenchDetails} of \appendixref{completeRes}.

\paragraph{Methodology}
For each benchmark, we measured the number of true-positive (TP), false-positive (FP) and false-negative (FN) results, and then calculated the \emph{precision} and \emph{recall} of each tool using the formulas below:
\begin{center}
\begin{tabular}{ccc}
$\text{\emph{Precision}} = \frac{TP}{TP+FP}$ & &
$\text{\emph{Recall}} = \frac{TP}{TP+FN}$ \\
\end{tabular}
\end{center}
High precision implies that a technique returns few irrelevant results, whereas  high recall implies that it misses only few relevant ones. 

Since ideal techniques have both high recall and high precision, the F-measure is commonly used to combine
both precision and recall into a single measure. The F-measure is defined as the harmonic mean of precision and recall, and is
calculated as follows:
\begin{center}
\begin{tabular}{c}
$\text{\emph{F-Measure}} = 2 \times \frac{Precision \times Recall}{Precision + Recall}$
\end{tabular}
\end{center}
The value of F-measure is high only when both precision and recall are high. We thus use the F-measure for accuracy evaluation.}

\begin{table}
%\setlength{\tabcolsep}{0.2em}
\begin{small}
\begin{center}
%\resizebox{\textwidth}{!}{
\begin{tabular}{|l||c|c|c|c|c|c|}
\hline
 & TPs & FPs & FNs & Precision & Recall & F-measure \\
\hline
\hline
 TaintDroid & 31  &  17  &  0  &  0.66  &  1.00  &  0.66 (0.78) \\
\hline
 \Tool & 29  &   1  &  2  &  0.98  &  0.97  &  0.96 (0.95) \\
 \hline
\end{tabular}
%\begin{tabular}{|c||C{1.4cm}|C{1.4cm}|C{1.4cm}|C{1.4cm}|C{1.4cm}|C{1.4cm}|}
% \hline
% & & & & & & \\[-0.08in]
% & TPs & FPs & FNs & Precision & Recall & F-measure \\[0.06in]
%\hline
%\hline
% & & & & & & \\[-0.08in]
% TaintDroid & 31  &  17  &  0  &  0.66  &  1.00  &  0.66 (0.78) \\[0.06in]
%\hline
% & & & & & & \\[-0.08in]
% \Tool & 29  &   1  &  2  &  0.98  &  0.97  &  0.96 (0.95) \\[0.06in]
%\hline
%\end{tabular}
%}% resizebox
 \end{center}
 \caption{\label{Ta:accuracyDBench}Accuracy of \Tool\ and TaintDroid on DroidBench}
\end{small}
\end{table}

\JC{\paragraph{Results}
The results obtained for both TaintDroid and \Tool\ on version 1.1 of DroidBench are summarized in \tableref{accuracyDBench} and presented in detail in \tableref{accuracyDBenchDetails}. The findings reported by \Tool\ are also publicly available.\footnote{
	See archive file
	\href{researcher.ibm.com/researcher/files/us-otripp/droidbench.zip}{researcher.ibm.com/researcher/files/us-otripp/droidbench.zip}.
}

Overall, TaintDroid detects 31 true leakages while also reporting 17 false positives, whereas our approach suffers from 2 false negatives, discovering 29 of the true leakages while flagging only 1 false alarm.
We averaged the F-measure calculation across all benchmarks (0.66 for TaintDroid vs 0.96 for \Tool). 
We also provide the ``global'' score computed by considering all TP, FP and FN results together in parentheses  (0.78 vs 0.95).

%We summarize the results as (i) the average across all app scores, shown outside the parentheses (0.96 for \Tool\ vs 0.69 for %TaintDroid), as well as (ii) the global accuracy score, computed as TPs / (TPs + FPs + Fns) (0.91 vs 0.65).

The results mark \Tool\ as visibly more accurate than TaintDroid. To further confirm this observation, we performed
Welch's T-Test: an adaptation of Student's T-Test intended to test the null hypothesis that the two population means (in our case, average F-measures) are equal. Welch's variant of the T-Test is used when two samples potentially have unequal variances~\cite{Welch:1947}.
The result of applying the test rejects the null hypothesis and demonstrates that \Tool\ performs significantly better than TaintDroid (p-value $\ll$ 0.05),
%The results mark \Tool\ as visibly more accurate than TaintDroid with respective accuracy scores of 0.96 vs 0.69, 
which confirms H1}.

\paragraph{Discussion}
Analysis of the per-benchmark findings reveals the following: 
First, the 2 false negatives of
\Tool\ on {\tt ImplicitFlow1} are both due to custom (i.e., non-standard) data transformations, which are outside the current scope of \Tool. An illustrative fragment from the {\tt ImplicitFlow1} code is shown in \figref{dataTransform}. The {\tt obfuscateIMEI($\ldots$)} transformation maps IMEI digits to English letters, which is a non-standard behavior that is unlikely to arise in an authentic app.

The false positive reported by \Tool, in common with TaintDroid, is on release of sensitive data to the file system, albeit using the {\tt MODE\_PRIVATE} flag, which does not constitute a leakage problem in itself. This can be resolved by performing Bayesian reasoning not only over argument values, but also over properties of the sink API (in this case, the storage location mapped to a file handle). We intend to implement this enhancement.

\begin{figure}
\begin{lstlisting}
TelephonyManager tm =
    getSystemService(TELEPHONY_SERVICE);
String imei = tm.getDeviceId(); //source
String obfuscatedIMEI = obfuscateIMEI(imei); ...;
Log.i(imei); // sink

private String obfuscateIMEI(String imei) {
  String result = "";
  for (char c : imei.toCharArray()) {
    switch(c) {
      case '0': result += 'a'; break;
      case '1': result += 'b'; break;
      case '2': result += 'c'; break; ...; } }
\end{lstlisting}
\caption{\label{Fi:dataTransform} Fragment from the DroidBench {\tt ImplicitFlow1} benchmark, which applies a custom transformation to private data}
\end{figure}


Beyond the false alarm in common with \Tool, TaintDroid has multiple other sources of imprecision. The main reasons for its false positives are
\begin{compactitem}
	\item coarse modeling of containers, mapping their entire contents to a single taint bit, which accounts e.g. for the false alarms on {\tt ArrayAccess\{1,2\}} and {\tt HashMapAccess1};
	\item field and object insensitivity, resulting in false alarms on {\tt FieldSensitivity\{2,4\}} and {\tt ObjectSensitivity\{1,2\}}; and more fundamentally,
	\item ignoring of data values, which causes TaintDroid to issue false warnings on {\tt LocationLeak\{1,2\}} even when location reads fail, yielding a {\tt Location} object without any meaningful information.
\end{compactitem}
The fundamental reason for these imprecisions is to constrain the overhead of TaintDroid, such that it can meet the performance demands of online privacy enforcement. \Tool\ is able to accommodate such optimizations while still ensuring high accuracy. 
\subsection{H2: Applicability}\label{Se:practical}
The second aspect of the evaluation compared between two versions of \Tool, whose sole difference lies in the method used for detecting relevant values: In one configuration (T-BD), relevant values are detected via tag propagation. The other configuration (H-BD) uses the heuristic detailed in \secref{points} of treating all values reachable from sink arguments \OC{(either directly or via the heap graph)} up to a depth bound of $k$ as relevant, which places more responsibility on Bayesian reasoning. We set $k$ at 3 based on manual review of the data structures flowing into privacy sinks. 

\JCO{We designed a parametric benchmark application to quantify the overhead reduction imposed by the H-BD variant of \Tool. The application} {First, to quantify overhead reduction, we designed a parametric benchmark application, which }
consists of a simple loop that flows the device IMEI into a log file.
Loop iterations perform intermediate data propagation steps. We then performed a series of experiments --- over the range of 1 to 19 propagation steps --- to quantify the relative overhead of tag propagation versus Bayesian analysis.

\begin{figure}
\includegraphics[width=\columnwidth]{OverheadDiagram.pdf}
\caption{\label{Fi:overhead}Overhead breakdown into tag propagation and Bayesian analysis at sink}
\end{figure}

The results, presented in \figref{overhead}, suggest that \JC{the overhead of} tag propagation is more dominant than \JC{that of} Bayesian analysis (with a ratio of roughly 2:1), even when the set of relevant values is naively over approximated. Discussion of the methodology underlying this experiment is provided in \appendixref{methodology}.

In general, H-BD trades overhead reduction for accuracy. 
H2 then asserts that, \JC{\emph{in practice}}, the tradeoff posed by H-BD is effective. \JC{Below, we discuss our empirical evaluation of this hypothesis over real-life subjects.}

\paragraph{Subjects} To avoid evaluators' bias, we applied the following selection process: We started from the 65 Google Play apps not chosen for the training phase. We then excluded 8 apps that do not have permission to access sensitive data and/or perform release operations (i.e., their manifest does not declare sufficient permissions out of {\tt INTERNET}, {\tt READ\_PHONE\_STATE}, {\tt SEND\_SMS}, etc), as well as 3 apps that we did not manage to install properly, resulting in 54 apps that installed successfully and exercise privacy sources and sinks.

\JC{
The complete list of the application we used is given in \tableref{realworldAll} of \appendixref{completeRes}. 
A subset of the applications, for which at least one leakage was detected, is also listed in \tableref{realworld}.

\paragraph{Methodology}
We deployed the apps under the two \Tool\ configurations.} Each execution was done from a clean starting state.
The third column of \JC{both} Tables~\ref{Ta:realworld} and \JC{\ref{Ta:realworldAll}} denotes whether our exploration of the app was exhaustive. By that we mean exercising all the UI points exposed by the app in a sensible order. Ideally we would do so for all apps. However, (i) some of the apps, and in particular gaming apps, had stability issues, and (ii) certain apps require SMS-validated sign in, which we did not perform. We did, however, create Facebook, Gmail and Dropbox accounts to log into apps that demand such information yet do not ask for SMS validation. We were also careful to execute the exact same crawling scenario under both the T-BD and H-BD configurations. We comment, from our experience, that most data leaks happen when an app launches, \OC{and initializes advertising/analytics functionality}, and so for apps for which deep crawling was not possible the results are still largely meaningful.

For comparability between the H-BD and T-BD configurations, we counted different dynamic reports involving the same pair of source/sink APIs as a single leakage instance.
We manually classified the findings into true positives and false positives. For this classification, we scrutinized the reports by the two configurations, and also --- in cases of uncertainty --- decompiled and/or reran the app to examine its behavior more closely. \JC{As in the experiment described in \secref{overhead}, we then calculated the precision, recall and F-measure for each of the tools.

\paragraph{Results}
The results obtained for H-BD and T-BD are summarized in \tableref{realworldAccuracy}. 
\tableref{realworld} summarizes the findings reported by both H-BD and T-BD at the granularity of privacy items: the device number, identifier and location, while \tableref{realworldAll} provides a detailed description of the results across all benchmarks including those on which no leakages were detected.
The warnings reported by the H-BD configuration are also publicly available for review.\footnote{
	See archive file realworldapps.zip available online at \href{https://www.dropbox.com/sh/ggrcvqsbkiubmlb/faSUXmr9xK}{https://www.dropbox.com/sh/
	ggrcvqsbkiubmlb/faSUXmr9xK}.
}}

\begin{table}
%\setlength{\tabcolsep}{0.2em}
\begin{small}
\begin{center}
%\resizebox{\textwidth}{!}{
%\begin{tabular}{|c||C{1.4cm}|C{1.4cm}|C{1.4cm}|C{1.4cm}|C{1.4cm}|C{1.4cm}|C{1.4cm}|}
% \hline
% & & & & & & & \\[-0.08in]
%      & TPs & FPs & FNs & Precision & Recall & F-measure & Crashes \\[0.06in]
%\hline
% & & & & & & & \\[-0.08in]
% H-BD  & 27  & 1    & 0   &  0.99 & 1.00 & 0.99 (0.98) & 12 \\[0.06in]
%\hline
% & & & & & & & \\[-0.08in]
% T-BD & 13    &  0   &   9      & 1.00 & 0.84 & 0.84 (0.74) & 22       \\[0.06in]
%\hline
%\end{tabular}
\begin{tabular}{|l||c|c|c|c|c|c|c|}
\hline
      & TPs & FPs & FNs & Precision & Recall & F-measure & Crashes \\
\hline
\hline
 H-BD  & 27  & 1    & 0   &  0.99 & 1.00 & 0.99 (0.98) & 12 \\
\hline
 T-BD & 13    &  0   &   9      & 1.00 & 0.84 & 0.84 (0.74) & 22 \\
 \hline
\end{tabular}
%}% resizebox
 \end{center}
 \caption{\label{Ta:realworldAccuracy}Accuracy of H-BD and T-BD \Tool\ configurations}
\end{small}
\end{table}


\begin{table*}
\begin{small}
\begin{center}
%\resizebox{\textwidth}{!}{
\begin{tabular}{|l|c|c||c|c|c||c|c|c|}
\hline
\multicolumn{1}{|c|}{\multirow{2}{*}{App}} & \multicolumn{1}{c|}{\multirow{2}{*}{Domain}} & \multicolumn{1}{c||}{\multirow{2}{*}{Deep crawl?}} & \multicolumn{3}{c||}{H-BD} & \multicolumn{3}{c|}{T-BD} \\
\cline{4-9}
                                          &                       &                & number       & dev. ID   &     location   &  number      & dev. ID   &     location \\
\hline
atsoft.games.smgame                    &       games/arcade       & \checkmark &            & \checkmark & \checkmark &            & \checkmark & \checkmark \\
com.antivirus                          &       communication      & \checkmark &            & \checkmark &            &            & \checkmark &            \\
{\bf com.appershopper.ios7lockscreen        }&{\bf       personalization    }&{\bf            }&{\bf \checkmark }&{\bf \checkmark }&{\bf \checkmark }&{\bf            }&{\bf            }&{\bf \checkmark }\\
com.bestcoolfungames.antsmasher        &       games/arcade       & \checkmark &            &            & \checkmark &            &            & \checkmark \\
{\bf com.bitfitlabs.fingerprint.lockscreen  }&{\bf       games/casual       }&{\bf            }&{\bf            }&{\bf \checkmark }&{\bf            }&{\bf            }&{\bf            }&{\bf            }\\
com.cleanmaster.mguard                 &       tools              & \checkmark &            & \checkmark &            &            & \checkmark &            \\
{\bf com.coolfish.cathairsalon              }&{\bf       games/casual       }&{\bf \checkmark }&{\bf            }&{\bf \checkmark }&{\bf            }&{\bf            }&{\bf            }&{\bf            }\\
{\bf com.coolfish.snipershooting            }&{\bf       games/action       }&{\bf \checkmark }&{\bf            }&{\bf \checkmark }&{\bf            }&{\bf            }&{\bf            }&{\bf            }\\
com.digisoft.TransparentScreen         &       entertainment      & \checkmark &            &            & \checkmark &            &            & \checkmark \\
com.g6677.android.cbaby                &       games/casual       &            &            & \checkmark &            &            &            &            \\
com.g6677.android.chospital            &       games/casual       &            &            & \checkmark &            &            &            &            \\
com.g6677.android.design               &       games/casual       &            &            & \checkmark &            &            &            &            \\
com.g6677.android.pnailspa             &       games/casual       &            &            & \checkmark &            &            &            &            \\
com.g6677.android.princesshs           &       games/casual       &            &            & \checkmark &            &            &            &            \\
com.goldtouch.mako                     &       news               & \checkmark &            & \checkmark &            &            & \checkmark &            \\
\hline
\hline
\multicolumn{1}{|c|}{15}               &                          &         8  &         1  &        13  &         4  &         0  &        4   &         4 \\
\hline
\end{tabular}
%}% resizebox
\end{center}
\caption{\label{Ta:realworld}Findings due to the H-BD and T-BD \Tool\ configurations on 15/54 top-popular mobile apps}
\end{small}
\end{table*}

%For \Tool\ we report two numbers: The one in parentheses is the restriction of the findings to sources and sinks in common with TaintDroid, whereas the outer number counts all findings by \Tool.

\JC{
As \tableref{realworldAccuracy} indicates,
%, and \tableref{realworldAll} affirms more explicitly, 
the H-BD variant is more accurate than the T-BD variant overall (F-measure of 0.99 vs. 0.84).
As in the experiment described in \secref{overhead}, we performed a Welch's T-Test on the F-Measure results to further confirm this observation. The test establishes that the difference between the accuracy of the tools is significant (p-value=0.003 $<$ 0.05).
Moreover, H-BD has a lower number of crashes and lower runtime overhead, which confirms H2.

\paragraph{Discussion}
To give the reader a taste of the findings, we present in \figrefs{ios7}{fruitninja} two examples of potential leakages that \Tool\ (both the H-BD and the T-BD configurations) deemed legitimate.} 
The instance in \figref{ios7} reflects the common scenario of obtaining the current (or last known) location, converting it into one or more addresses, and then releasing only the country or zip code. In the second instance, in \figref{fruitninja}, the 64-bit Android ID --- generated when the user first sets up the device --- is read via a call to \texttt{Settings$\$$Secure.getString(ANDROID\_ID)}. At the release point, into the file system, only a prefix of the Android ID consisting of the first 12 digits is published.

\begin{figure}
\begin{lstlisting}[numbers=none]
$\fbox{source : private value}$
    GeoCoder.getFromLocation(...) : [ Lat: ..., Long: ...,
	    Alt: ..., Bearing: ..., ..., $\textbf{IL}$ ]
	    
$\fbox{sink : arguments}$
  WebView.loadUrl(...) : http://linux.appwiz.com/
    profile/72/72_exitad.html?
    p1=RnVsbCtBbmRyb2lkK29uK0VtdWxhdG9y&
    p2=Y2RmMTUxMjRlYTRjN2FkNQ%3d%3d&
    ... LOCATION=$\textbf{IL}$& ...
    MOBILE_COUNTRY_CODE=&
    NETWORK=WIFI
\end{lstlisting}
\caption{\label{Fi:ios7}Suppressed warning on ios7lockscreen}
\end{figure}

\begin{figure}
\begin{lstlisting}[numbers=none]
$\fbox{source : private value}$
  Settings$\$$Secure.getString(...) : $\textbf{cdf15124ea4c7ad5}$
  
$\fbox{sink : arguments}$
  FileOutputStream.write(...) :
    <?xml version='1.0' encoding='utf-8'
    standalone='yes'
    ?><map><string
    name="openudid">$\textbf{cdf15124ea4c}$
\end{lstlisting}
\caption{\label{Fi:fruitninja}Suppressed warning on fruitninjafree}
\end{figure}

\JC{
As \tableref{realworld} makes apparent, the findings by H-BD are more complete:
It detects 27 leakages (versus 13 reports by T-BD), with no false-negative results and only one false positive.
We attribute that to (i) the intrusive instrumentation required for tag propagation, which can cause instabilities, and (ii) inability to track tags through native code, as discussed below.
}

The T-BD variant introduces significantly more instability than the H-BD variant, causing illegal application behaviors in 21 cases compared to only 12 under H-BD. We have investigated this large gap between the H-BD and T-BD configurations, including by decompiling the subject apps. Our analysis links the vast majority of illegal behaviors to limitations that
TaintDroid casts on loading of third-party libraries. For this reason, certain functionality is not executed, also leading to exceptional app states, which both inhibit certain data leaks.\footnote{
	For a technical explanation, see forum comment by William Enck, the TaintDroid moderator, at \href{https://groups.google.com/forum/\#!topic/android-security-discuss/U1fteEX26bk}{https://groups.google.com/forum/\#!topic/android-security-discuss/U1fteEX26bk}.
}

A secondary reason why H-BD is able to detect more leakages, e.g. in the lockscreen app, is that this benchmark makes use of the {\tt mobileCore} module,\footnote{
\href{https://www.mobilecore.com/sdk/}{https://www.mobilecore.com/sdk/}
} which is a highly optimized and obfuscated library. We suspect that data-flow tracking breaks within this library, though we could not fully confirm this.

At the same time, the loss in accuracy due to heuristic identification of relevant values is negligible, as suggested by the discussion in \secref{points}. H-BD triggers only one false alarm, on  ios7lockscreen,
which is due to overlap between irrelevant values: extra information on the {\tt Location} object returned by a call to {\tt LocationManager.getLastKnownLocation(...)} and unrelated metadata passed into a {\tt ContextWrapper.startService(...)} request.  Finally, as expected, H-BD does not incur false negatives.




%In the following, we discuss the accuracy, coverage and robustness trends that emerge from \tableref{realworld}.
%
%\paragraph{Accuracy} First, \Tool\ produces more complete results, triggering $>$x2.5 more alarms (21 vs 8) on $>$x2 of the benchmarks (15 vs 7) when restricted to the APIs supported by TaintDroid, and $>$x8 alarms (65 vs 8) on $\sim$x4 of the benchmarks (26 vs 7) in general. The results by both tools are highly precise. We identified only one false alarm --- by \Tool --- due to overlap between benign values: extra information on the {\tt Location} object returned by a call to {\tt LocationManager.getLastKnownLocation(...)} and unrelated metadata passed into a {\tt ContextWrapper.startService(...)} request.
%
%There are no findings reported by TaintDroid that \Tool\ fails to detect. There are, however, alarms triggered by \Tool\ over source/sink pairs supported by TaintDroid that TaintDroid misses
%(e.g., on {\tt ios7lockscreen}, {\tt fingerprint} and {\tt princesshs}). We invested considerable time investigating into these findings, including decompiling the subject apps. For some of the apps (like {\tt princesshs}), the reason for the misses by TaintDroid appears to be that
%TaintDroid limits loading of third-party libraries, and thus certain functionality is not executed, also leading to exceptional app states, which both inhibit certain data leaks.\footnote{
%	See \href{https://groups.google.com/forum/\#!topic/android-security-discuss/U1fteEX26bk}{forum comment} on this by William Enck, the TaintDroid moderator.
%} As for the remaining apps, these all leak data via the {\tt mobileCore} module,\footnote{
%\href{https://www.mobilecore.com/sdk/}{https://www.mobilecore.com/sdk/}
%} which is a highly optimized and obfuscated library, and so a plausible explanation for the misses is that data flow breaks within this library, though we could not confirm this.
%
%A pleasing property of the data-centric approach, especially in comparison with taint analysis, is that it can record not only true alarms but also candidate leakages falling below the designated similarity threshold. We thus went through an extended log of our system, containing also suppressed alarms, and confirmed that all the source/sink pairs that the system resolved not to report were indeed benign.
%
%\paragraph{Coverage} Our second observation is that the TaintDroid APIs are insufficient. The additional internet, system-settings and file-system APIs monitored by our system disclose many more data leaks. We expect that extending TaintDroid to include these additional APIs while retaining the same level of precision would not be easy. (See section 8 in \cite{EGCCJMS:OSDI10}.) We also anticipate observable impact on the performance of TaintDroid.
%
%\paragraph{Robustness} A final point is that \Tool\ appears more stable than TaintDroid, crashing on 8 of the 54 applications, which are a strict --- and small --- subset of the 21 apps TaintDroid crashes on. \Tool\ crashes all stem from missing libraries and/or hardware capabilities in the emulated environment. We conjecture that additional TaintDroid crashes are due to (i) TaintDroid restrictions on loading of third-party libraries, as well as (ii) the intrusive nature of the TaintDroid instrumentation scheme, which reaches into the lowest levels of the Android platform, thereby affecting its stability. This analysis suggests that our approach is more readily applicable to real-world applications.
%
%
%
%
%The tainting approach has been investigated, and shown to be feasible, across multiple studies. A notable implementation of the tainting approach is the TaintDroid system~\cite{EGCCJMS:OSDI10}. TaintDroid is the product of careful engineering, balancing between precision and performance to achieve admissible runtime overhead (mostly below 10\%). TaintDroid tracks taint labels across a select set of sources and sinks. It supports system-wide analysis by tainting files and interprocess messages. TaintDroid has been extended by, and used in, multiple recent studies~\cite{JHW:SPSM12,RCE:CODAPSY13,SPKS:WISTP12}.
