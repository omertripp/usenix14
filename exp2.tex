\subsection{H2: Applicability}\label{Se:practical}
The second aspect of the evaluation compared between two versions of \Tool, whose sole difference lies in the method used for detecting relevant values: In one configuration (T-BD), relevant values are detected via tag propagation. The other configuration (H-BD) uses the heuristic detailed in \secref{points} of treating all values reachable from sink arguments \OC{(either directly or via the heap graph)} up to a depth bound of $k$ as relevant, which places more responsibility on Bayesian reasoning. We set $k$ at 3 based on manual review of the data structures flowing into privacy sinks. 

\JCO{We designed a parametric benchmark application to quantify the overhead reduction imposed by the H-BD variant of \Tool. The application} {First, to quantify overhead reduction, we designed a parametric benchmark application, which }
consists of a simple loop that flows the device IMEI into a log file.
Loop iterations perform intermediate data propagation steps. We then performed a series of experiments --- over the range of 1 to 19 propagation steps --- to quantify the relative overhead of tag propagation versus Bayesian analysis.

\begin{figure}
\includegraphics[width=\columnwidth]{OverheadDiagram.pdf}
\caption{\label{Fi:overhead}Overhead breakdown into tag propagation and Bayesian analysis at sink}
\end{figure}

The results, presented in \figref{overhead}, suggest that \JC{the overhead of} tag propagation is more dominant than \JC{that of} Bayesian analysis (with a ratio of roughly 2:1), even when the set of relevant values is naively over approximated. Discussion of the methodology underlying this experiment is provided in \appendixref{methodology}.

In general, H-BD trades overhead reduction for accuracy. 
H2 then asserts that, \JC{\emph{in practice}}, the tradeoff posed by H-BD is effective. \JC{Below, we discuss our empirical evaluation of this hypothesis over real-life subjects.}

\paragraph{Subjects} To avoid evaluators' bias, we applied the following selection process: We started from the 65 Google Play apps not chosen for the training phase. We then excluded 8 apps that do not have permission to access sensitive data and/or perform release operations (i.e., their manifest does not declare sufficient permissions out of {\tt INTERNET}, {\tt READ\_PHONE\_STATE}, {\tt SEND\_SMS}, etc), as well as 3 apps that we did not manage to install properly, resulting in 54 apps that installed successfully and exercise privacy sources and sinks.

\JC{
The complete list of the application we used is given in \tableref{realworldAll} of \appendixref{completeRes}. 
A subset of the applications, for which at least one leakage was detected, is also listed in \tableref{realworld}.

\paragraph{Methodology}
We deployed the apps under the two \Tool\ configurations.} Each execution was done from a clean starting state.
The third column of \JC{both} Tables~\ref{Ta:realworld} and \JC{\ref{Ta:realworldAll}} denotes whether our exploration of the app was exhaustive. By that we mean exercising all the UI points exposed by the app in a sensible order. Ideally we would do so for all apps. However, (i) some of the apps, and in particular gaming apps, had stability issues, and (ii) certain apps require SMS-validated sign in, which we did not perform. We did, however, create Facebook, Gmail and Dropbox accounts to log into apps that demand such information yet do not ask for SMS validation. We were also careful to execute the exact same crawling scenario under both the T-BD and H-BD configurations. We comment, from our experience, that most data leaks happen when an app launches, \OC{and initializes advertising/analytics functionality}, and so for apps for which deep crawling was not possible the results are still largely meaningful.

For comparability between the H-BD and T-BD configurations, we counted different dynamic reports involving the same pair of source/sink APIs as a single leakage instance.
We manually classified the findings into true positives and false positives. For this classification, we scrutinized the reports by the two configurations, and also --- in cases of uncertainty --- decompiled and/or reran the app to examine its behavior more closely. \JC{As in the experiment described in \secref{overhead}, we then calculated the precision, recall and F-measure for each of the tools.

\paragraph{Results}
The results obtained for H-BD and T-BD are summarized in \tableref{realworldAccuracy}. 
\tableref{realworld} summarizes the findings reported by both H-BD and T-BD at the granularity of privacy items: the device number, identifier and location, while \tableref{realworldAll} provides a detailed description of the results across all benchmarks including those on which no leakages were detected.
The warnings reported by the H-BD configuration are also publicly available for review.\footnote{
	See archive file realworldapps.zip available online at \href{https://www.dropbox.com/sh/ggrcvqsbkiubmlb/faSUXmr9xK}{https://www.dropbox.com/sh/
	ggrcvqsbkiubmlb/faSUXmr9xK}.
}}

\begin{table}
%\setlength{\tabcolsep}{0.2em}
\begin{small}
\begin{center}
%\resizebox{\textwidth}{!}{
%\begin{tabular}{|c||C{1.4cm}|C{1.4cm}|C{1.4cm}|C{1.4cm}|C{1.4cm}|C{1.4cm}|C{1.4cm}|}
% \hline
% & & & & & & & \\[-0.08in]
%      & TPs & FPs & FNs & Precision & Recall & F-measure & Crashes \\[0.06in]
%\hline
% & & & & & & & \\[-0.08in]
% H-BD  & 27  & 1    & 0   &  0.99 & 1.00 & 0.99 (0.98) & 12 \\[0.06in]
%\hline
% & & & & & & & \\[-0.08in]
% T-BD & 13    &  0   &   9      & 1.00 & 0.84 & 0.84 (0.74) & 22       \\[0.06in]
%\hline
%\end{tabular}
\begin{tabular}{|l||c|c|c|c|c|c|c|}
\hline
      & TPs & FPs & FNs & Precision & Recall & F-measure & Crashes \\
\hline
\hline
 H-BD  & 27  & 1    & 0   &  0.99 & 1.00 & 0.99 (0.98) & 12 \\
\hline
 T-BD & 13    &  0   &   9      & 1.00 & 0.84 & 0.84 (0.74) & 22 \\
 \hline
\end{tabular}
%}% resizebox
 \end{center}
 \caption{\label{Ta:realworldAccuracy}Accuracy of H-BD and T-BD \Tool\ configurations}
\end{small}
\end{table}


\begin{table*}
\begin{small}
\begin{center}
%\resizebox{\textwidth}{!}{
\begin{tabular}{|l|c|c||c|c|c||c|c|c|}
\hline
\multicolumn{1}{|c|}{\multirow{2}{*}{App}} & \multicolumn{1}{c|}{\multirow{2}{*}{Domain}} & \multicolumn{1}{c||}{\multirow{2}{*}{Deep crawl?}} & \multicolumn{3}{c||}{H-BD} & \multicolumn{3}{c|}{T-BD} \\
\cline{4-9}
                                          &                       &                & number       & dev. ID   &     location   &  number      & dev. ID   &     location \\
\hline
atsoft.games.smgame                    &       games/arcade       & \checkmark &            & \checkmark & \checkmark &            & \checkmark & \checkmark \\
com.antivirus                          &       communication      & \checkmark &            & \checkmark &            &            & \checkmark &            \\
{\bf com.appershopper.ios7lockscreen        }&{\bf       personalization    }&{\bf            }&{\bf \checkmark }&{\bf \checkmark }&{\bf \checkmark }&{\bf            }&{\bf            }&{\bf \checkmark }\\
com.bestcoolfungames.antsmasher        &       games/arcade       & \checkmark &            &            & \checkmark &            &            & \checkmark \\
{\bf com.bitfitlabs.fingerprint.lockscreen  }&{\bf       games/casual       }&{\bf            }&{\bf            }&{\bf \checkmark }&{\bf            }&{\bf            }&{\bf            }&{\bf            }\\
com.cleanmaster.mguard                 &       tools              & \checkmark &            & \checkmark &            &            & \checkmark &            \\
{\bf com.coolfish.cathairsalon              }&{\bf       games/casual       }&{\bf \checkmark }&{\bf            }&{\bf \checkmark }&{\bf            }&{\bf            }&{\bf            }&{\bf            }\\
{\bf com.coolfish.snipershooting            }&{\bf       games/action       }&{\bf \checkmark }&{\bf            }&{\bf \checkmark }&{\bf            }&{\bf            }&{\bf            }&{\bf            }\\
com.digisoft.TransparentScreen         &       entertainment      & \checkmark &            &            & \checkmark &            &            & \checkmark \\
com.g6677.android.cbaby                &       games/casual       &            &            & \checkmark &            &            &            &            \\
com.g6677.android.chospital            &       games/casual       &            &            & \checkmark &            &            &            &            \\
com.g6677.android.design               &       games/casual       &            &            & \checkmark &            &            &            &            \\
com.g6677.android.pnailspa             &       games/casual       &            &            & \checkmark &            &            &            &            \\
com.g6677.android.princesshs           &       games/casual       &            &            & \checkmark &            &            &            &            \\
com.goldtouch.mako                     &       news               & \checkmark &            & \checkmark &            &            & \checkmark &            \\
\hline
\hline
\multicolumn{1}{|c|}{15}               &                          &         8  &         1  &        13  &         4  &         0  &        4   &         4 \\
\hline
\end{tabular}
%}% resizebox
\end{center}
\caption{\label{Ta:realworld}Findings due to the H-BD and T-BD \Tool\ configurations on 15/54 top-popular mobile apps}
\end{small}
\end{table*}

%For \Tool\ we report two numbers: The one in parentheses is the restriction of the findings to sources and sinks in common with TaintDroid, whereas the outer number counts all findings by \Tool.

\JC{
As \tableref{realworldAccuracy} indicates,
%, and \tableref{realworldAll} affirms more explicitly, 
the H-BD variant is more accurate than the T-BD variant overall (F-measure of 0.99 vs. 0.84).
As in the experiment described in \secref{overhead}, we performed a Welch's T-Test on the F-Measure results to further confirm this observation. The test establishes that the difference between the accuracy of the tools is significant (p-value=0.003 $<$ 0.05).
Moreover, H-BD has a lower number of crashes and lower runtime overhead, which confirms H2.

\paragraph{Discussion}
To give the reader a taste of the findings, we present in \figrefs{ios7}{fruitninja} two examples of potential leakages that \Tool\ (both the H-BD and the T-BD configurations) deemed legitimate.} 
The instance in \figref{ios7} reflects the common scenario of obtaining the current (or last known) location, converting it into one or more addresses, and then releasing only the country or zip code. In the second instance, in \figref{fruitninja}, the 64-bit Android ID --- generated when the user first sets up the device --- is read via a call to \texttt{Settings$\$$Secure.getString(ANDROID\_ID)}. At the release point, into the file system, only a prefix of the Android ID consisting of the first 12 digits is published.

\begin{figure}
\begin{lstlisting}[numbers=none]
$\fbox{source : private value}$
    GeoCoder.getFromLocation(...) : [ Lat: ..., Long: ...,
	    Alt: ..., Bearing: ..., ..., $\textbf{IL}$ ]
	    
$\fbox{sink : arguments}$
  WebView.loadUrl(...) : http://linux.appwiz.com/
    profile/72/72_exitad.html?
    p1=RnVsbCtBbmRyb2lkK29uK0VtdWxhdG9y&
    p2=Y2RmMTUxMjRlYTRjN2FkNQ%3d%3d&
    ... LOCATION=$\textbf{IL}$& ...
    MOBILE_COUNTRY_CODE=&
    NETWORK=WIFI
\end{lstlisting}
\caption{\label{Fi:ios7}Suppressed warning on ios7lockscreen}
\end{figure}

\begin{figure}
\begin{lstlisting}[numbers=none]
$\fbox{source : private value}$
  Settings$\$$Secure.getString(...) : $\textbf{cdf15124ea4c7ad5}$
  
$\fbox{sink : arguments}$
  FileOutputStream.write(...) :
    <?xml version='1.0' encoding='utf-8'
    standalone='yes'
    ?><map><string
    name="openudid">$\textbf{cdf15124ea4c}$
\end{lstlisting}
\caption{\label{Fi:fruitninja}Suppressed warning on fruitninjafree}
\end{figure}

\JC{
As \tableref{realworld} makes apparent, the findings by H-BD are more complete:
It detects 27 leakages (versus 13 reports by T-BD), with no false-negative results and only one false positive.
We attribute that to (i) the intrusive instrumentation required for tag propagation, which can cause instabilities, and (ii) inability to track tags through native code, as discussed below.
}

The T-BD variant introduces significantly more instability than the H-BD variant, causing illegal application behaviors in 21 cases compared to only 12 under H-BD. We have investigated this large gap between the H-BD and T-BD configurations, including by decompiling the subject apps. Our analysis links the vast majority of illegal behaviors to limitations that
TaintDroid casts on loading of third-party libraries. For this reason, certain functionality is not executed, also leading to exceptional app states, which both inhibit certain data leaks.\footnote{
	For a technical explanation, see forum comment by William Enck, the TaintDroid moderator, at \href{https://groups.google.com/forum/\#!topic/android-security-discuss/U1fteEX26bk}{https://groups.google.com/forum/\#!topic/android-security-discuss/U1fteEX26bk}.
}

A secondary reason why H-BD is able to detect more leakages, e.g. in the lockscreen app, is that this benchmark makes use of the {\tt mobileCore} module,\footnote{
\href{https://www.mobilecore.com/sdk/}{https://www.mobilecore.com/sdk/}
} which is a highly optimized and obfuscated library. We suspect that data-flow tracking breaks within this library, though we could not fully confirm this.

At the same time, the loss in accuracy due to heuristic identification of relevant values is negligible, as suggested by the discussion in \secref{points}. H-BD triggers only one false alarm, on  ios7lockscreen,
which is due to overlap between irrelevant values: extra information on the {\tt Location} object returned by a call to {\tt LocationManager.getLastKnownLocation(...)} and unrelated metadata passed into a {\tt ContextWrapper.startService(...)} request.  Finally, as expected, H-BD does not incur false negatives.